%!TEX TS-program = sage+XeLaTeX
\documentclass[12pt]{article}
\usepackage[letterpaper,twoside,margin=1cm,bindingoffset=1cm,]{geometry}

%\usepackage{graphicx}
\usepackage{amsmath}
\usepackage{amssymb}
\usepackage{eso-pic}
\usepackage{pgf,pgffor}
\usepackage{tikz}
\usepackage{sagetex}
\nofiles

\usepackage{fontspec,xltxtra,xunicode}
\defaultfontfeatures{Mapping=tex-text}
\setromanfont[Mapping=tex-text]{Gentium Book Basic}
\setsansfont[Scale=MatchLowercase,Mapping=tex-text]{Josefin Sans}
\setmonofont[Scale=MatchLowercase]{Consolas}

%%%%%%%%%%%%%% 
% Put your info here
\newcommand\Names{Emmy N\"oether, Maria Goeppert, Chien-Shiung Wu}
\newcommand\Field{Quiz on Calculating with Units, Unit Conversions}
\newcommand\Course{Natural Science}
\newcommand\Date{Marchtember 41, 2301} % use \today for auto-dating
%%%%%%%%%%%%%%

\pagestyle{empty}

\begin{document}
% Numerical context for Wind Turbine problem
\begin{sagesilent}
 v=0
 P=0
 P_in_kW=0
 def reset_windturbine():
     global v
     global P
     global P_in_kW
     global P_display
     v = RealDistribution('uniform', [8,22]).get_random_element()
     P = 0.2*v^3
     P_in_kW = round(P/1000,2)
     P = round(P_in_kW * 1000)
     v = round((P/.2)^(1/3),1) 	
     return ''
\end{sagesilent}

% Create different version of exam for each student:
\foreach \name in \Names {%
% Graph paper in background
% Letter page size is 215.9 mm x 279.4 mm
% The grid doesn't align properly, currently.
% TODO: Fix grid to be more intuitive
% TODO: Store plots once. This creates each tikzpicture, sage calculation for each student, which can take a long time for 130 students.
\begin{tikzpicture}[remember picture, overlay, shift=(current page.south west)]
	\draw[step=1cm,lightgray,thin] (-2cm,-26cm) grid ++(20cm,27cm);
\end{tikzpicture}%

% Exam Heading
{\sffamily %
	\noindent {\large \name } \hfill \Date
	
	\noindent \Field \hfill \Course}
\vspace{1cm}

\noindent Problem:
% Easier to start backward with wind speed in mph
% Note that including comments with # in the code below will lead to a pgffor error.
% Placing sage verbatim environment within pgffor argument results in errors.  One can add ^^J to the end of each line or put the results in a LaTeX savebox

\sagestr{reset_windturbine()}Students study the wind turbine they have selected to mount on the top of their school. Between average wind speeds of 8~mph and 22~mph, the Skystream 3.7's power output is proportional to the cube of the steady wind speed and can be modeled by the equation $P=(0.2\,\text{W mph}^{-3}) v^3$, where $P$ represents the power in watts (W) and $v$ represents the steady wind speed in miles per hour (mph). At what wind speed will the wind turbine produce \sage{P_in_kW} kilowatts of electricity?

\newpage 
% Solution Heading
{\sffamily %
	\noindent {\large \name } \hfill \Date
	
	\noindent Solutions to \Field \hfill \Course}
\vspace{1cm}

\sageplot[width=.9\textwidth]{plot(0.2*x^3, x, 8, 22, title='Electrical Power Output For Skystream 3.7 v. Steady Wind Speed', axes_labels=['$v$ / mph','$P$ / W'], gridlines=True)}

\begin{align*}
P					&=(0.2\,\text{W mph}^{-3}) v^3 \\
\sage{P_in_kW}\,\text{kW}  	&=(0.2\,\text{W mph}^{-3}) v^3 \\
\sage{P}\,\text{W}  	&=(0.2\,\text{W mph}^{-3}) v^3 \\
\sage{round(P/0.2)}\,\text{mph}^3 	&=v^3 \\
\sage{v}\,\text{mph}	&\approx v
\end{align*}
As a check on our work, notice that \sage{v}~mph is between 8~mph and 22~mph, when our model is valid, so we conclude that a wind blowing steadily at about \sage{v}~mph would generate \sage{P_in_kW}~kW of electricity.

\vfill
\eject
\newpage
} % end foreach

\end{document}